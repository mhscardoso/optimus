\documentclass{beamer}

\usepackage[brazil]{babel}
\usepackage[utf8]{inputenc}
\usepackage{fancybox}

\newcommand{\fone}{\sum_{i=0}^{5} \left(100(x_{i+1} - x_{i}^{2})^2 + (1 - x_i)^2\right)}
\newcommand{\ftwo}{\sum_{i=0}^{99} x_i^4 - 16 x_i^2 + 5 x_i}
\newcommand{\fthree}{(x_1^2 + x_2 - 11)^2 + (x_1 + x_2^2 - 7)^2}
\newcommand{\R}{\mathbb{R}}

\usetheme{Pittsburgh}
\usecolortheme{owl}

\mode<presentation> { \setbeamercovered{transparent} }
\setbeamertemplate{navigation symbols}{}
\makeatletter
\beamer@centeredfalse
\def\beamerorig@set@color{%
  \pdfliteral{\current@color}%
  \aftergroup\reset@color
}
\def\beamerorig@reset@color{\pdfliteral{\current@color}}


\title[Otimização]
{Algoritmos de Otimização}

\subtitle{Aplicação de Algoritmos em uma função específica}

\author[Matheus Cardoso]
{M. Cardoso\inst{1}}

\institute[UFRJ]
{
  \inst{1}%
  Engenharia de Computação e Informação\\
  Universidade Federal do Rio de Janeiro
}

\date[2024]
{Dezembro de 2024}

\begin{document}


\frame{\titlepage}


\begin{frame}
\frametitle{As funções}

\begin{align*}
	\text{min} && f_1(x) &= \fone \\
	\text{s.a.} && x \in \R^7
\end{align*}

\begin{align*}
	\text{min} && f_2(x) &= \ftwo \\
	\text{s.a.} && x \in \R^{100}
\end{align*}

\begin{align*}
	\text{min} && f_3(x) &= \fthree \\
	\text{s.a.} && x \in \R^2
\end{align*}

\end{frame}


\begin{frame}
\frametitle{Analisando a função $f_1$}

\begin{align*}
	f_1(x) &= \fone
\end{align*} \pause

Percebe-se que:
\[
	\left(100(x_{i+1} - x_{i}^{2})^2 + (1 - x_i)^2\right) \geq 0
\]

Pois $(10(x_{i+1} - x_{i}^{2}))^2 \geq 0$ e 
$(1 - x_i)^2 \geq 0$. \pause

Mínimo quando ambas são zero. \pause

Na segunda parcela, é preciso que $x_i$ seja igual a 1 ($1 - x_i = 0$). \pause

Consequentemente, $x_{i+1} = 1$ para zerar a primeira parcela. \pause

O $x$ que garante o mínimo será $x^{T} = (1 \ 1 \ 1 \ 1 \ 1 \ 1 \ 1)$

\end{frame}


\begin{frame}
\frametitle{Analisando a função $f_2$}

\begin{align*}
	f_2(x) &= \ftwo
\end{align*} \pause

Como a função:
\[
	x_i^4 - 16 x_i^2 + 5 x_i
\]
já possui mínimo, o mínimo de $f_2$ será um vetor cujos valores
serão o $x$ que garante o mínimo para $x^4 - 16 x^2 + 5 x$ \pause

Plotemos
\end{frame}

\begin{frame}
\frametitle{Analisando a função $f_2$}

\end{frame}

\end{document}
